\newpage
{\color{gray}\hrule}
\begin{center}
\section{Introduction}
\bigskip
\end{center}
{\color{gray}\hrule}

Given a markdown file at the beginning of the project by class professor Dr. Coyle, team members used the example code in this markdown file to complete the project with some slight modifications to fit needs. This code, naturally in Python in an Advanced Python course, required packages for the spacy, networkx, matplotlib.pyplot, FAISS, and langchain modules. In addition to the modules required in the example code, members added the tiktoken module for token control. In addition to installing modules, to use the system developed for this report a user must also include an OpenAI API key in the environment for use of their gpt-3.5 model.\par
Members chose the \textit{A Song of Fire and Ice} due to a specific member's familiarity to the series and due to its ease of availability online in the ".pdf" format. These books include: \bigskip

\begin{enumerate}[label=\arabic*.]
    \item \textbf{A Game of Thrones} (1996)
    \item \textbf{A Clash of Kings} (1998)
    \item \textbf{A Storm of Swords} (2000)
    \item \textbf{A Feast for Crows} (2005)
    \item \textbf{A Dance with Dragons} (2011)
\end{enumerate}
\bigskip

These books must be in ".txt" format in the same directory as all other components in order for the developed system to function, so ".pdf" files need to be converted prior to programming the model. Furthermore, due to bandwidth limitations, only two books could be selected at a time for chunking per instance.
\bigskip